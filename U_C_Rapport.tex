\documentclass[a4paper, 12pt]{article}
% Importation du préambule dans un .tex séparé

\usepackage[utf8]{inputenc}
\usepackage[a4paper, margin=18mm]{geometry}
\usepackage{amsmath}
\usepackage{enumitem,amssymb}
\usepackage{graphicx}
\usepackage{subfig}
\usepackage{float}
\usepackage{pdfpages}
\usepackage{textcomp, gensymb}
\usepackage{comment}
\usepackage{wrapfig}
\usepackage[export]{adjustbox}
\usepackage{hyperref}
\usepackage{multirow}
\usepackage{multicol}
\usepackage{color}
\usepackage{booktabs}
\usepackage{import}
\usepackage{longtable}
\usepackage{hyperref,wasysym}
\usepackage{csquotes}
\usepackage{siunitx}
\usepackage[style=ieee]{biblatex}
\usepackage{tikz}
\usepackage{pgfgantt}
\usepackage{xcolor}
\usepackage{tocbibind} 
\usepackage[T1]{fontenc} 
\usepackage[french]{babel}
\usepackage{appendix}

\addbibresource{Bibliographie/reference.bib} %Import the bibliography file


\newlist{todolist}{itemize}{2}
\setlist[todolist]{label=$\square$}

\setlength{\parindent}{0pt}

\definecolor{Jaune}{RGB}{240, 211, 34}
\definecolor{RAL6038}{RGB}{0, 181, 27}
\definecolor{barblue}{RGB}{153,204,254}
\definecolor{groupblue}{RGB}{51,102,254}
\definecolor{linkred}{RGB}{165,0,33} 

\renewcommand*\contentsname{Table des matières}
\renewcommand{\listfigurename}{Liste des figures}
\renewcommand{\listtablename}{Liste des tableaux}
%\renewcommand{\abstract}{Résumé}
\graphicspath{ {./Graphiques/} }
\date{\today}
\begin{document}
% Importation de la Page de Garde dans un .tex séparé

%%% PAGE DE TITRE %%%
\begin{titlepage}
%\vskip-12.73mm
\title{\vspace{-12.73mm}
%\centering\includegraphics[scale=0.5]{Pdg/Logos_Hepia.png}
\includegraphics[height=10mm]{Pdg/logo_hepia.pdf}\hfill \includegraphics[height=10mm]{Pdg/logo_hes-so.pdf}\\%%\rule{\textwidth}{0.5pt}
    %%\centering\includegraphics[scale=0.5]{Logos\_Hepia.png}\\
    \centering\rule{17cm}{0.1mm}\vspace*{0.4in}\\
   Titre normal \\
   \textbf{Titre en gras}}
\author{NOM Prénom$^1$
        \vspace*{0.2in}\\
        \small$^1$hepia, Classe, prénom.nom@etu.hesge.ch}

\maketitle
\begin{center}
\rule{17cm}{0.1mm}
\end{center}
\begin{abstract}
        Le résumé
\end{abstract}
\end{titlepage}
\tableofcontents
\newpage
\listoffigures
\listoftables
\newpage
\section{Figures}
%%%%%%%%%% EXAMPLE POUR L'INCLUSION DE FIGURES
\section{Figure simple}
\begin{figure}[H]
    \centering
    \includegraphics[width=\textwidth]{filename.eps} 
    \caption{Légende de la Figure simple}\label{fig:filename1}
\end{figure}
\section{Figures côtes à côtes}
\begin{figure}[H]
    % On crée un environnement figure dans lequel on met des minipages    
    \begin{minipage}{0.49\textwidth} % Minipage gauche
        \centering
        \includegraphics[width=\textwidth]{filename.eps} 
        \caption{Légende de la figure de gauche}\label{fig:filename2}
    \end{minipage}
    \hfill
    \begin{minipage}{0.49\textwidth} % Minipage droite
        \centering
        \includegraphics[width=\textwidth]{Laser_L_I.eps}
        \caption{Légende de la figure de droite}\label{fig:Laser_L_I1}
    \end{minipage}
\end{figure}
\section{Figures triples} 
\begin{figure}[H]
    % On crée un environnement figure dans lequel on met des minipages    
    \begin{minipage}{0.49\textwidth} % Minipage gauche
        \centering
        \includegraphics[width=\textwidth]{filename.eps} 
        \caption{Légende du filename.eps}\label{fig:filename3}
    \end{minipage}
    \hfill
    \begin{minipage}{0.49\textwidth} % Minipage droite
        \centering
        \includegraphics[width=\textwidth]{Laser_L_I.eps}
        \caption{Légende du Laser\_L\_I.eps}\label{fig:Laser_L_I2}
    \end{minipage}
    \centering
    \includegraphics[width=0.49\textwidth]{Laser_P_I.eps}
    \caption{Légende du Laser\_PI\_.eps}\label{fig:Laser_P_I}
    %\caption{Légende Générale}
\end{figure}
La Figure filename.eps et sa référence~\ref{fig:filename3}
\newpage
\section{Formules}
\subsection{Matrices}
%%%%%%%%%% EXAMPLE POUR DES MATRICES
\begin{align}
    s\begin{pmatrix}x_1\\x_2\\x_3\end{pmatrix} = 
    \begin{pmatrix}
        -\frac{1}{T_{ele}} & 0 & 0 \\
        \frac{K_{mot}}{J} & -\frac{b_{mot}}{J} & 0\\
        0 & i_{reduc} & 0 
    \end{pmatrix}
    \cdot
    \begin{pmatrix}
    x_1\\
    x_2\\
    x_3
    \end{pmatrix}
    +
    \begin{pmatrix}
        \frac{K_{ele}}{T_{ele}}\\
        0\\
        0
    \end{pmatrix}
    w
\end{align}
\newpage
%%%%%%%%%% EXAMPLE POUR DES TABLEAUX  
\section{Tableaux}

%%%%%%%%%% EXAMPLE POUR L'INCLUSION DE LA BIBLIOGRAPHIE
\nocite{*} % Permet d'afficher toutes les entrées de la biblio sans citation dans le texte
\printbibliography[title={Références}] % Affiche la bibliographie avec le titre Références
\end{document}
